\section{Implementation}
\label{section:implementation}
The N-Queens hill climbing solution presented in this paper was implemented in AngularJS, in order to more easily
develop the solution, and quickly prototype a UI built for exploring the problem. The solution presented here is an
excellent source for learning Hill Climbing because of certain extensions made to the original assignment, and the
focus on interaction.

\subsection{Application Structure}
The application itself is stored as AngularJS code served by a Node.js API. The project is built with modern practices,
so \texttt{package.json} describes the Node.js dependencies and project configuration and \texttt{bower.json} describes
the client-side dependencies.

\subsubsection{Setup}
To run the project, the following must be installed on the machine:
\texttt{node}\cite{node}, \texttt{npm}\cite{npm}, \texttt{bower}\cite{bower}.

Then use \texttt{npm} and \texttt{bower} to install the dependencies:

\begin{lstlisting}
# Install server dependencies
npm install

# Install client JS dependencies
bower install
\end{lstlisting}

\subsubsection{Running}
The project is setup to run easily, using scripts defined in \texttt{package.json}. Simply run the following:

\begin{lstlisting}
npm start
\end{lstlisting}

Then browse to the page at \url{http://localhost:8080}

\subsubsection{Javascript Files}
The core of the project is in the Javascript files under \texttt{n-queens/public/js}. Here, each folder and file is
described:


